\RequirePackage{luatex85}
\documentclass[10p,a4paper,reqno,titlepage]{report}


%scrreprt
%----------------------------------------------------------
% This is a sample document for the AMS LaTeX Book or Monograph Class
% Class options
%       --  Body text point size:
%  %                        8pt, 9pt, 10pt (default), 11pt, 12pt
%       --  Paper size:  letterpaper (8.5x11 inch, default), a4paper
%       --  Orientation: portrait(default), landscape
%       --  Print side:  oneside, twoside (default)
%       --  Quality:     final(default), draft
%       --  Title page:  titlepage, notitlepage
%       --  Start chapter on left:
%                        openright (no, default), openany
%       --  Columns:     onecolumn (default), twocolumn
%       --  Omit extra math features:
%                        nomath
%       --  AMS fonts (noamasfonts available):
%                        noamsfonts
%       --  PSAMSfonts (fewer AMSfontsizes)
%                        psamsfonts
%       --  Equation numbering (equation numbers on the left is the default)
%                        leqno (default), reqno
%       --  Equation centering (equations centered is the default)
%                        centeredtags (default}, tbtags (top, bottom)
%       --  Displayed equations (centered is the default)
%                        fleqn (flush left)
% For instance the command
%          \documentclass[a4paper,12p,reqno]{amsbook}
% ensures that the paper size is a4, fonts are typeset at the size 12p
% and the equation numbers are on the right side.
\usepackage{../allpack}
\usepackage[a4paper,hmargin={2cm,2cm},twoside,includehead=true,headsep=2mm,tmargin=2cm,bmargin=2cm,headheight=15pt]{geometry}



\title{Théorie des valeurs extrêmes \\N. Kazi-Tani \\
M2 PSA\\
Semestre S9}
%\author{Dader}
\date{}
\begin{document}

%	\pagestyle{fancy}
%	\lhead{Probabilités, semestre S6}
%	\rhead{}
%	\chead{}
\maketitle
\tableofcontents
\newpage

\chapter*{Introduction}
L'un des buts de la théorie des valeurs extrêmes est d'extrapoler une probabilité de survie à partir des données collectées (par exemple les sinistres et leurs coûts pour les assurances/réassurances).

On peut par exemple s'intéresser à une suite de variables aléatoires indépendantes et identiquement distribuées $(X_k)_{k\geq 1}$ et se poser la question de la loi suivie par la suite $(M_n = \max\{X_k, 1 \leq k \leq n\})_{n\geq 1}$ pour $n$ assez grand.

On peut par exemple se poser les questions suivantes:\
\begin{enumerate}
	\item Quelle est la vitesse minimale d'un 100 mètres?
	\item Quelle est l'estimation des sinistres extrêmes en assurance ou en réassurance?
	\item Quels sont les mouvements extrêmes des prix des actifs sur un marché financier?
	\item \`A quelle hauteur doit-on placer des digues pour contrer des inondations décennale, centennale...?
	\item Quel estimation des temps de services d'un réseau télécom?
\end{enumerate}
Toutes ces questions peuvent se traiter comme l'étude de queues de distributions, ce qui fait intervenir la théorie des valeurs extrêmes.

\begingroup
\let\clearpage\relax
\bibliographystyle{plain}
\bibliography{ve_bib.bib}
\endgroup

\end{document}